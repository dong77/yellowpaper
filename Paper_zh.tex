% !TEX program = XeLaTeX
% !TEX encoding = UTF-8
\documentclass[UTF8,nofonts]{ctexart}
\setCJKmainfont[BoldFont=FandolSong-Bold.otf,ItalicFont=FandolKai-Regular.otf]{FandolSong-Regular.otf}
\setCJKsansfont[BoldFont=FandolHei-Bold.otf]{FandolHei-Regular.otf}
\setCJKmonofont{FandolFang-Regular.otf}

%\usepackage{tweaklist}
\usepackage{url}
\usepackage{cancel}
\usepackage{xspace}
\usepackage{graphicx}
\usepackage{multicol}
\usepackage{subfig}
\usepackage{amsmath}
\usepackage{amssymb}
\usepackage[a4paper,width=170mm,top=18mm,bottom=22mm,includeheadfoot]{geometry}
\usepackage{booktabs}
\usepackage{array}
\usepackage{verbatim}
\usepackage{caption}
\usepackage{natbib}
\usepackage{float}
\usepackage{pdflscape}
\usepackage{mathtools}
\usepackage[usenames,dvipsnames]{xcolor}
\usepackage{afterpage}
\usepackage{tikz}

\newcommand{\hcancel}[1]{%
    \tikz[baseline=(tocancel.base)]{
        \node[inner sep=0pt,outer sep=0pt] (tocancel) {#1};
        \draw[black] (tocancel.south west) -- (tocancel.north east);
    }%
}%

\definecolor{lightyellow}{rgb}{1,0.98,0.9}
\definecolor{lightpink}{rgb}{1,0.94,0.95}

\newcommand{\firsthomesteadblock}{\ensuremath{N_H}}

\DeclarePairedDelimiter{\ceil}{\lceil}{\rceil}
\newcommand*\eg{e.g.\@\xspace}
\newcommand*\Eg{e.g.\@\xspace}
\newcommand*\ie{i.e.\@\xspace}
%\renewcommand{\itemhook}{\setlength{\topsep}{0pt}  \setlength{\itemsep}{0pt}\setlength{\leftmargin}{15pt}}

\title{Ethereum: A Secure Decentralised Generalised Transaction Ledger \\ {\smaller \textbf{Homestead revision}}}
\author{
    Dr. Gavin Wood\\
    Founder, Ethereum \& Ethcore\\
    gavin@ethcore.io
}

\begin{document}

%\pagecolor{lightyellow}
\pagecolor{lightpink}

\begin{abstract}
一些列的项目,特别是比特币,已经证明了区块链在结合了经密码保护的交易后的应用价值。每个这样项目都可以被视为建立在去中心化的,整体而言却构成单例的,计算资源上的简单应用。我们可以称区块链的模式为共享状态,支持交易的单例计算机。

以太坊用一种更通用的方式实现这种模式。而且它提供丰富的计算资源,每个资源都有不同的状态和操作代码,但能够通过消息传递框架与彼此交互。在本文我们探讨它的设计,实现问题,以及它带来的机遇和未来可能的挑战。
\end{abstract}

\setlength{\columnsep}{20pt}
\begin{multicols}{2}

\section{介绍}\label{sec:introduction}

With ubiquitous internet connections in most places of the world, global information transmission has become incredibly cheap. Technology-rooted movements like Bitcoin have demonstrated, through the power of the default, consensus mechanisms and voluntary respect of the social contract that it is possible to use the internet to make a decentralised value-transfer system, shared across the world and virtually free to use. This system can be said to be a very specialised version of a cryptographically secure, transaction-based state machine. Follow-up systems such as Namecoin adapted this original ``currency application'' of the technology into other applications albeit rather simplistic ones.

Ethereum is a project which attempts to build the generalised technology; technology on which all transaction-based state machine concepts may be built. Moreover it aims to provide to the end-developer a tightly integrated end-to-end system for building software on a hitherto unexplored compute paradigm in the mainstream: a trustful object messaging compute framework.

\subsection{Driving Factors} \label{ch:driving}

There are many goals of this project; one key goal is to facilitate transactions between consenting individuals who would otherwise have no means to trust one another. This may be due to geographical separation, interfacing difficulty, or perhaps the incompatibility, incompetence, unwillingness, expense, uncertainty, inconvenience or corruption of existing legal systems. By specifying a state-change system through a rich and unambiguous language, and furthermore architecting a system such that we can reasonably expect that an agreement will be thus enforced autonomously, we can provide a means to this end.

Dealings in this proposed system would have several attributes not often found in the real world. The incorruptibility of judgement, often difficult to find, comes naturally from a disinterested algorithmic interpreter. Transparency, or being able to see exactly how a state or judgement came about through the transaction log and rules or instructional codes, never happens perfectly in human-based systems since natural language is necessarily vague, information is often lacking, and plain old prejudices are difficult to shake.

Overall, I wish to provide a system such that users can be guaranteed that no matter with which other individuals, systems or organisations they interact, they can do so with absolute confidence in the possible outcomes and how those outcomes might come about.

\subsection{Previous Work} \label{ch:previous}

\cite{buterin2013ethereum} first proposed the kernel of this work in late November, 2013. Though now evolved in many ways, the key functionality of a block-chain with a Turing-complete language and an effectively unlimited inter-transaction storage capability remains unchanged.

Hashcash, introduced by \cite{back2002hashcash} (in a five-year retrospective), provided the first work into the usage of a cryptographic proof of computational expenditure as a means of transmitting a value signal over the Internet. Though not widely adopted, the work was later utilised and expanded upon by \cite{nakamoto2008bitcoin} in order to devise a cryptographically secure mechanism for coming to a decentralised social consensus over the order and contents of a series of cryptographically signed financial transactions. The fruits of this project, Bitcoin, provided a first glimpse into a decentralised transaction ledger.

Other projects built on Bitcoin's success; the alt-coins introduced numerous other currencies through alteration to the protocol. Some of the best known are Litecoin and Primecoin, discussed by \cite{sprankel2013technical}. Other projects sought to take the core value content mechanism of the protocol and repurpose it; \cite{aron2012bitcoin} discusses, for example, the Namecoin project which aims to provide a decentralised name-resolution system.

Other projects still aim to build upon the Bitcoin network itself, leveraging the large amount of value placed in the system and the vast amount of computation that goes into the consensus mechanism. The Mastercoin project, first proposed by \cite{mastercoin2013willett}, aims to build a richer protocol involving many additional high-level features on top of the Bitcoin protocol through utilisation of a number of auxiliary parts to the core protocol. The Coloured Coins project, proposed by \cite{colouredcoins2012rosenfeld}, takes a similar but more simplified strategy, embellishing the rules of a transaction in order to break the fungibility of Bitcoin's base currency and allow the creation and tracking of tokens through a special ``chroma-wallet''-protocol-aware piece of software.

Additional work has been done in the area with discarding the decentralisation foundation; Ripple, discussed by \cite{boutellier2014pirates}, has sought to create a ``federated'' system for currency exchange, effectively creating a new financial clearing system. It has demonstrated that high efficiency gains can be made if the decentralisation premise is discarded.

Early work on smart contracts has been done by \cite{szabo1997formalizing} and \cite{miller1997future}. Around the 1990s it became clear that algorithmic enforcement of agreements could become a significant force in human cooperation. Though no specific system was proposed to implement such a system, it was proposed that the future of law would be heavily affected by such systems. In this light, Ethereum may be seen as a general implementation of such a \textit{crypto-law} system.

%E language?

\end{multicols}
\end{document}