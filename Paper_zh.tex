% !TEX program = XeLaTeX
% !TEX encoding = UTF-8
\documentclass[UTF8,nofonts]{ctexart}
\setCJKmainfont[BoldFont=FandolSong-Bold.otf,ItalicFont=FandolKai-Regular.otf]{FandolSong-Regular.otf}
\setCJKsansfont[BoldFont=FandolHei-Bold.otf]{FandolHei-Regular.otf}
\setCJKmonofont{FandolFang-Regular.otf}

%\usepackage{tweaklist}
\usepackage{url}
\usepackage{cancel}
\usepackage{xspace}
\usepackage{graphicx}
\usepackage{multicol}
\usepackage{subfig}
\usepackage{amsmath}
\usepackage{amssymb}
\usepackage[a4paper,width=180mm,top=18mm,bottom=22mm,includeheadfoot]{geometry}
\usepackage{booktabs}
\usepackage{array}
\usepackage{verbatim}
\usepackage{caption}
\usepackage{natbib}
\usepackage{float}
\usepackage{pdflscape}
\usepackage{mathtools}
\usepackage[usenames,dvipsnames]{xcolor}
\usepackage{afterpage}
\usepackage{tikz}

\newcommand{\hcancel}[1]{%
    \tikz[baseline=(tocancel.base)]{
        \node[inner sep=0pt,outer sep=0pt] (tocancel) {#1};
        \draw[black] (tocancel.south west) -- (tocancel.north east);
    }%
}%

\definecolor{lightyellow}{rgb}{1,0.98,0.9}
\definecolor{lightpink}{rgb}{1,0.94,0.95}

\newcommand{\firsthomesteadblock}{\ensuremath{N_H}}

\DeclarePairedDelimiter{\ceil}{\lceil}{\rceil}
\newcommand*\eg{e.g.\@\xspace}
\newcommand*\Eg{e.g.\@\xspace}
\newcommand*\ie{i.e.\@\xspace}
%\renewcommand{\itemhook}{\setlength{\topsep}{0pt}  \setlength{\itemsep}{0pt}\setlength{\leftmargin}{15pt}}

\title{Ethereum: A Secure Decentralised Generalised Transaction Ledger \\ {\smaller \textbf{Homestead revision}}}
\author{
    Dr. Gavin Wood\\
    Founder, Ethereum \& Ethcore\\
    gavin@ethcore.io
}
\begin{document}

%\pagecolor{lightyellow}
\pagecolor{lightpink}

\begin{abstract}
一些列的项目,特别是比特币,已经证明了区块链模式在结合了经密码保护的交易后的应用价值。每个这样项目都可以被视为建立在去中心化的,整体而言却构成单例的,计算资源上的简单应用。我们可以称区块链的模式为共享状态,支持交易的单例计算机。

以太坊用一种更通用的方式实现这种模式。而且它提供丰富的计算资源,每个资源都有不同的状态和操作代码,但能够通过消息传递框架与彼此交互。在本文我们探讨它的设计,实现问题,以及它带来的机遇和未来可能的挑战。
\end{abstract}

\setlength{\columnsep}{20pt}
\begin{multicols}{2}

\section{介绍}\label{sec:introduction}

With ubiquitous internet connections in most places of the world, global information transmission has become incredibly cheap. Technology-rooted movements like Bitcoin have demonstrated, through the power of the default, consensus mechanisms and voluntary respect of the social contract that it is possible to use the internet to make a decentralised value-transfer system, shared across the world and virtually free to use. This system can be said to be a very specialised version of a cryptographically secure, transaction-based state machine. Follow-up systems such as Namecoin adapted this original ``currency application'' of the technology into other applications albeit rather simplistic ones.

Ethereum is a project which attempts to build the generalised technology; technology on which all transaction-based state machine concepts may be built. Moreover it aims to provide to the end-developer a tightly integrated end-to-end system for building software on a hitherto unexplored compute paradigm in the mainstream: a trustful object messaging compute framework.

\subsection{Driving Factors} \label{ch:driving}

There are many goals of this project; one key goal is to facilitate transactions between consenting individuals who would otherwise have no means to trust one another. This may be due to geographical separation, interfacing difficulty, or perhaps the incompatibility, incompetence, unwillingness, expense, uncertainty, inconvenience or corruption of existing legal systems. By specifying a state-change system through a rich and unambiguous language, and furthermore architecting a system such that we can reasonably expect that an agreement will be thus enforced autonomously, we can provide a means to this end.

Dealings in this proposed system would have several attributes not often found in the real world. The incorruptibility of judgement, often difficult to find, comes naturally from a disinterested algorithmic interpreter. Transparency, or being able to see exactly how a state or judgement came about through the transaction log and rules or instructional codes, never happens perfectly in human-based systems since natural language is necessarily vague, information is often lacking, and plain old prejudices are difficult to shake.

Overall, I wish to provide a system such that users can be guaranteed that no matter with which other individuals, systems or organisations they interact, they can do so with absolute confidence in the possible outcomes and how those outcomes might come about.

\subsection{Previous Work} \label{ch:previous}

\cite{buterin2013ethereum} first proposed the kernel of this work in late November, 2013. Though now evolved in many ways, the key functionality of a block-chain with a Turing-complete language and an effectively unlimited inter-transaction storage capability remains unchanged.

Hashcash, introduced by \cite{back2002hashcash} (in a five-year retrospective), provided the first work into the usage of a cryptographic proof of computational expenditure as a means of transmitting a value signal over the Internet. Though not widely adopted, the work was later utilised and expanded upon by \cite{nakamoto2008bitcoin} in order to devise a cryptographically secure mechanism for coming to a decentralised social consensus over the order and contents of a series of cryptographically signed financial transactions. The fruits of this project, Bitcoin, provided a first glimpse into a decentralised transaction ledger.

Other projects built on Bitcoin's success; the alt-coins introduced numerous other currencies through alteration to the protocol. Some of the best known are Litecoin and Primecoin, discussed by \cite{sprankel2013technical}. Other projects sought to take the core value content mechanism of the protocol and repurpose it; \cite{aron2012bitcoin} discusses, for example, the Namecoin project which aims to provide a decentralised name-resolution system.

Other projects still aim to build upon the Bitcoin network itself, leveraging the large amount of value placed in the system and the vast amount of computation that goes into the consensus mechanism. The Mastercoin project, first proposed by \cite{mastercoin2013willett}, aims to build a richer protocol involving many additional high-level features on top of the Bitcoin protocol through utilisation of a number of auxiliary parts to the core protocol. The Coloured Coins project, proposed by \cite{colouredcoins2012rosenfeld}, takes a similar but more simplified strategy, embellishing the rules of a transaction in order to break the fungibility of Bitcoin's base currency and allow the creation and tracking of tokens through a special ``chroma-wallet''-protocol-aware piece of software.

Additional work has been done in the area with discarding the decentralisation foundation; Ripple, discussed by \cite{boutellier2014pirates}, has sought to create a ``federated'' system for currency exchange, effectively creating a new financial clearing system. It has demonstrated that high efficiency gains can be made if the decentralisation premise is discarded.

Early work on smart contracts has been done by \cite{szabo1997formalizing} and \cite{miller1997future}. Around the 1990s it became clear that algorithmic enforcement of agreements could become a significant force in human cooperation. Though no specific system was proposed to implement such a system, it was proposed that the future of law would be heavily affected by such systems. In this light, Ethereum may be seen as a general implementation of such a \textit{crypto-law} system.

%-------------------------------------
\section{区块链(Blockchain)模式} \label{ch:overview}

以太坊作为一个整体可被视为基于交易的状态机:我们从一个创世纪(genesis)状态开始,递增执行交易来改变状态机到某个最终状态。这个最终状态将被我们认定为以太坊世界的权威性``版本''。状态中包含的信息包括账户余额,声誉,信任安排,物理世界数据,终止,任何计算机可以表示的数据都可以。因此交易是两个状态间的一条合法的弧线;这里`合法`非常重要 -- 不合法的变更改变要远远多于合法的状态改变。 例如减少一个账户的余额但没有在其它账户中做等量余额增加就是不合法的状态变更。一个合法的状态需要经过一个交易。形式化表达为:

\begin{equation}
\boldsymbol{\sigma}_{t+1} \equiv \Upsilon(\boldsymbol{\sigma}_t, T)
\end{equation}

其中$\Upsilon$是以太坊状态转换函数。 在以太坊中,$\Upsilon$ 和$\boldsymbol{\sigma}$ 结合一起比任何现存的比较系统都更强大。$\Upsilon$ 允许模块进行任意计算,而 $\boldsymbol{\sigma}$ 允许模块存储交易间的任何状态。

多个交易被收集合并成一个区块(Block),多个区块通过基于密码学的散列值链接起来成为区块链,散列值也被用作块的引用。区块的作用类似与日志,用来记录一些列交易,前一个区块,以及最终状态的标识符(但不存储最终状态本身 --- 否者就太大了)。区块间穿插着节点\textit{挖矿}所需的激励。这种激励措施在状态转换函数中发生,其结果是给指定的账户增加余额。

挖矿是通过贡献算力(工作量)推举一系列交易(一个区块)战胜其它潜在的竞争区块的过程。挖矿依赖于一种密码学上安全的证明。这种证明被称为工作量证明(Proof-of-Work),我们会在 \ref{ch:pow} 章节做详细探讨。

我们在公式上进一步展开:
\begin{eqnarray}
\boldsymbol{\sigma}_{t+1} & \equiv & \Pi(\boldsymbol{\sigma}_t, B) \\
B & \equiv & (..., (T_0, T_1, ...) ) \\
\Pi(\boldsymbol{\sigma}, B) & \equiv & \Omega(B, \Upsilon(\Upsilon(\boldsymbol{\sigma}, T_0), T_1) ...)
\end{eqnarray}

其中 $\Omega$ 是确认区块终态的变更函数(它回馈指定的参与方);$B$ 是当前区块,包含一些列交易和其它数据;$\Pi$ 是区块级的状态变更函数。

这就是区块链的基础模型,它不仅仅是以太坊,也是目前所有去中心化的,基于共识的交易系统的骨架。


\subsection{价值}

为了激励网络中的计算,以太坊需要有个公认的价值传递方式。为了解决这个问题,以太坊引入了一个内在的货币,以太(Ether),也被称作{\small ETH},有时用英文字母 \DH{}表示。以太的最小单位,也是该货币整数值计数所用的单位,叫伟(Wei)。一个以太被定义为 $10^{18}$ 个伟。除此之外还有其它以子单位:

\par
\begin{center}
\begin{tabular}{rl}
\toprule
乘数 & 子单位名字 \\
\midrule
$10^0$ & 伟(Wei) \\
$10^{12}$ & 萨博(Szabo) \\
$10^{15}$ & 芬尼(Finney) \\
$10^{18}$ & 以太(Ether) \\
\bottomrule
\end{tabular}
\end{center}
\par


在本文中提到以太价值,货币,余额,付款的时候,我们默认都是以伟作为计数单位。

\subsection{哪个历史?}

既然系统是去中心化的,且任何参与方都有机会在前序区块的基础上生成一个新的区块,其结果就一定是个区块的树形结构。为了对区块链,也就是从根节点区块(创世纪区块)到某个叶节点区块(包含最新交易的区块)的路径,达成共识,就需要一个大家都认同的方案。如果网络节点对哪个路径才是`最好'的区块链存在分歧,就会发生\textit{fork} 。

这意味着在某个时间节点(区块)后,可能有多种系统状态共存:一些节点认为这个区块包含权威的交易,另一些节点认为那个区块包含权威的交易,这些区块可能包含根本不同或不兼容的交易。这种情况需要尽最大可能避免,否则造成的不确定性可能毁掉对整个系统的信心。

我们用来达成共识的方案是由\cite{cryptoeprint:2013:881}引入的GHOST协议的一个简化版。我们在\ref{ch:ghost}章节描述这个方案的流程。

%-------------------------------------

\section{约定}\label{ch:conventions}

我使用一系列的印刷体上的约定来做符号表达,其中一些是本文特有的:

两类高度结构化的,`顶级的'状态值,用粗体小写希腊字母表示。这两类分别是世界状态,用$\boldsymbol{\sigma}$(或其变种)表示,和机器状态,用$\boldsymbol{\mu}$表示。

高度结构化值上的函数用大写希腊字母表示,比如以太坊状态转换函数用$\Upsilon$表示。

大多数函数用大写字母表示,比如一般的成本函数(cost function)用$C$表示。函数符号可以通过加下标来表示特定的变种,比如{\tiny SSTORE} 操作的成本函数用$C_\text{\tiny SSTORE}$表示。对于特定的和有可能是外部定义的函数,我可能用打字机字体表示,比如Keccak-256散列函数(SHA-3竞赛的获奖者)用$\texttt{KEC}$表示(通常被称为简单Keccak散列函数)。$\texttt{KEC512}$被称为Keccak 512散列函数。

元组(Tuple)一般用一个大写字母表示,比如$T$用来表示一个以太坊交易。可为该符号加下标来标识它的单个组成部分,比如$T_n$表示该交易的nonce。下标的格式标明其类型,比如大写的下标说明该元组有可通过下标标识组成部分。

标量(Scalar)和固定大小的字节序列(或叫字节数组)用一个小写字母表示,比如本文中交易的nonce用$n$表示。有特殊意义的值可能用希腊字母表示,比如栈上某个操作所需的数据数目用$\delta$表示。

任意长度序列一般用一个粗体小写字母表示,比如消息调用的输出是个字节序列,它用$\mathbf{o}$表示。特别重要的值可能用一个粗体大写字母表示。

我们通篇假设标量都是正整数,因此属于$\mathbb{P}$集合。所有字节顺序列的集合为 $\mathbb{B}$,在附录\ref{app:rlp}中给出正式定义。有特别长度限制的字符序列集合用下标标识,因此所有长度为$32$ 的字节序列集合被称为$\mathbb{B}_{32}$,所有小于$2^{256}$ 的正整数集合被称为$\mathbb{P}_{256}$。其正式定义被放到\ref{ch:block}章节。

方括号用来索引和引用序列中的单个元素或子序列,比如 $\boldsymbol{\mu}_\mathbf{s}[0]$表示机器栈上的第一项。对于子序列,我们用省略号来指明区域,比如$\boldsymbol{\mu}_\mathbf{m}[0..31]$表示机器内存的前32项。

对于全局状态$\boldsymbol{\sigma}$,其本身是一组账户,账户本身又是元组,方括号用来表示单个账户。


当考虑现有值的各种变化的时候,我的原则是在定义的范围内,如果用占位符$\Box$ 表示某个未被更改的`输入'值,那么改变后和可用值就用$\Box'$表示,中间值用$\Box^*$,  $\Box^{**}$等表示。特别情况下,为了最大化可读性并且在不引入歧义时候,我可能用字母或数字下标去标识中间值,尤其是那些需要特别关注的中间值。

在使用现有函数时,对于函数 $f$,我用函数$f^*$表示一个和原函数相似的,但是是在序列元素级别的映射函数。其正式定义被放到\ref{ch:block}章节。

另外我定义了一些有用的函数。其中一个常用的是 $\ell$,它返回一个序列的末项。


\begin{equation}
\ell(\mathbf{x}) \equiv \mathbf{x}[\lVert \mathbf{x} \rVert - 1]
\end{equation}


%-------------------------------------

\section{区块, 状态和交易} \label{ch:bst}

介绍完以太坊的基本概念后,我们来更细地讨论交易,区块,和状态的含义。


\subsection{世界状态(World State)} \label{ch:state}

世界状态 (\textit{状态})是地址(160位标识符)和账户状态(序列化成RLP的一种数据结构,参见附录\ref{app:rlp})之间的映射表。 虽然世界状态并不存储在区块链中,但会被存储在一个经过改造的Merkle Patricia树(\textit{Trie树}, 参见附录 \ref{app:trie})中。Trie树需要一个简单的后台数据库来维护从字节数组(bytearray)到字节数组的一个映射表;我们称这个底层的数据库为状态数据库。使用状态数据库有一系列优势:首先该结构的根节点在密码学上依赖于所有内部数据,因此根节点的散列值可被用作整个系统状态的一个安全的身份标识。其次,该结构是不可改变的,因此可以通过适当更改根节点的散列值回退到任何之前的(散列值已知的)状态。因为我们在区块链上存储所有根节点的散列值,这样就可以毫不费力地做状态回退。

账户状态包含以下四个域:

\begin{description}
\item[nonce] 临造数,是一个标量,其值是从该地址发送出的交易数;对于有代码的账户,其值是该账户生成的合约数量。对于在状态$\boldsymbol{\sigma}$中的地址$a$,该域可形式上表示为 $\boldsymbol{\sigma}[a]_n$。
\item[balance] 余额,是一个标量,其值为该地址拥有的以太数量(以`伟'为单位)。形式上表示为$\boldsymbol{\sigma}[a]_b$。
\item[storageRoot] 存储根散列,它是存储账户内容的Merkle Patricia树的根节点的256位散列值。账户内容是一个从256位整数的键(key)的256位Keccak散列值到RLP编码的256位整数的值(value)的映射表,编码成一个Trie树形式。存储散列形式上表示为 $\boldsymbol{\sigma}[a]_s$。
\item[codeHash] 代码散列,该账户EVM代码 --- 就是该地址一旦接收到消息调用就要执行的代码 --- 的散列值;由于账户代码是不可更改的,因此不像其它域,该域在账户生成后就不能被更改。所有代码片段都存储在状态数据库中,并且可用代码散列查询使用。代码散列形式上表示为 $\boldsymbol{\sigma}[a]_c$,因此代码如果表示为$\mathbf{b}$,就有$\texttt{\small KEC}(\mathbf{b}) = \boldsymbol{\sigma}[a]_c$。
\end{description}

我一般会提及Trie树中存储的键值对,而不是数根节点的散列值,因此我给出下面这个方便使用的等价定义:

\begin{equation}
\texttt{\small TRIE}\big(L_I^*(\boldsymbol{\sigma}[a]_\mathbf{s})\big) \equiv \boldsymbol{\sigma}[a]_s
\end{equation}

$L_I^*$是Trie树中键值对集合整体的坍塌函数,它被定义为在每个键值对上应用基本函数$L_I$的结果。$L_I$可被标示为:

\begin{equation}
L_I\big( (k, v) \big) \equiv \big(\texttt{\small KEC}(k), \texttt{\small RLP}(v)\big)
\end{equation}

其中:
\begin{equation}
k \in \mathbb{B}_{32} \quad \wedge \quad v \in \mathbb{P}
\end{equation}

需要明白的一点是,$\boldsymbol{\sigma}[a]_\mathbf{s}$并不是该账户的`物理'成员,因此对账户的序列化没有影响。【译者:$\boldsymbol{\sigma}[a]_\mathbf{s}$对于每个区块只有一个,因此被该块中的多个账户所共享。】

如果 \textbf{codeHash} 域是空字符串的256位Keccak散列值,即: $\boldsymbol{\sigma}[a]_c = \texttt{\small KEC}\big(()\big)$,那么该账户就是简单账户,有时也叫`非合约'账户。

因此我们可以定义一个世界状态的坍塌函数$L_S$:

\begin{equation}
L_S(\boldsymbol{\sigma}) \equiv \{ p(a): \boldsymbol{\sigma}[a] \neq \varnothing \}
\end{equation}

其中:
\begin{equation}
p(a) \equiv  \big(\texttt{\small KEC}(a), \texttt{\small RLP}\big( (\boldsymbol{\sigma}[a]_n, \boldsymbol{\sigma}[a]_b, \boldsymbol{\sigma}[a]_s, \boldsymbol{\sigma}[a]_c) \big) \big)
\end{equation}

函数$L_S$ 和Trie树函数一起被用来提供一个世界状态的短标识符(散列值)。我们假设:

\begin{equation}
\forall a: \boldsymbol{\sigma}[a] = \varnothing \; \vee \; (a \in \mathbb{B}_{20} \; \wedge \; v(\boldsymbol{\sigma}[a]))
\end{equation}

其中 $v$是账户的验证函数:

\begin{equation}
\quad v(x) \equiv x_n \in \mathbb{P}_{256} \wedge x_b \in \mathbb{P}_{256} \wedge x_s \in \mathbb{B}_{32} \wedge x_c \in \mathbb{B}_{32}
\end{equation}



\subsection{家园(Homestead)} \label{ch:homestead}

界定平台从 {\it 前沿(Frontier)}阶段 到 {\it 家园(Homestead)}阶段间过度的那个区块对于公共网络上的兼容性至关重要。我们把它的编号\firsthomesteadblock 定义为:
\begin{equation}
\firsthomesteadblock \equiv 1,\! 150,\! 000
\end{equation}

从这个区块起,平台的协议被更新。为了体现这种协议变更,该符号会在一些公式中出现。

\subsection{交易} \label{ch:transaction}

交易 (用$T$表示) 是以太坊外的某个角色创建的,经过密码学签名的指令。虽然最终的角色总是人,但还是会通过软件来做指令的生成和传播\footnote{Notably, such `tools' could ultimately become so causally removed from their human-based initiation---or humans may become so causally-neutral---that there could be a point at which they rightly be considered autonomous agents. \eg contracts may offer bounties to humans for being sent transactions to initiate their execution.}。存在两种不同的交易:一种交易的执行结果是消息调用,另一种交易的执行结果是生成附有代码的新账户(也叫`合约生成')。两种交易都包括下面几个公共域:

\begin{description}
\item[nonce] 临造数,是一个标量,它的值是发送者发出的交易数,用$T_n$表示。
\item[gasPrice] 油费价格,是一个标量。交易执行过程中需要为所有计算支付\textit{油费(gas)}。这个标量的值是每单位gas需要花费的以太数量。该值的单位是Wei,用$T_p$表示。
\item[gasLimit] 油费上限,是一个标量,代表该交易执行过程中可消耗的gas的最大值。该费用是需要在交易执行前预支付的,且后续无法增大,用$T_g$表示。
\item[to] 消息调用的长度为160位的接收地址;如果交易是用来生成新的合约,该值为$\varnothing$(这里代表$\mathbb{B}_0$的唯一成员)。该值用$T_t$表示。
\item[value]代表转给目标地址的以太数。如果交易是用来生成新的合约,该值代表捐赠给该新合约的以太数。该值的单位是Wei,用$T_v$表示。
\item[v, r, s] 与交易签名相关的三个域,可用作来确定交易的发送者。表示为$T_w$, $T_r$ 和 $T_s$。附录 \ref{app:signing}中会对该域进行展开。
\end{description}

除此之外,生成合约的交易还包含:

\begin{description}
\item[init] 用来存储账户初始化所需EVM代码的一个大小不限的字节数组,表示为$T_\mathbf{i}$。
\end{description}

\textbf{init}是一段EVM代码;它会返回另一段叫\textbf{body}的代码。\textbf{body}在该账户每次接收到消息调用(被交易触发或因内部代码调用)时都会被执行一次。因此\textbf{init}只有在账户创建的时候执行一次,之后便被立即丢弃。

相比而言,一个消息调用交易包含:

\begin{description}
\item[data] 用来存储消息调用输入数据的一个大小不限的字节数组,表示为$T_\mathbf{d}$。
\end{description}
在附录 \ref{app:signing} 中我们会具体说明一个用交易去计算其发送者的函数$S$, 实现方式是用SECP-256k1曲线上的ECDSA方法,对交易(除去用于签名的三个域)的哈希值进行签名计算。 目前我们简单用$S(T)$表示交易$T$的发送者。
\begin{equation}
L_T(T) \equiv \begin{cases}
(T_n, T_p, T_g, T_t, T_v, T_\mathbf{i}, T_w, T_r, T_s) & \text{如} \; T_t = \varnothing\\
(T_n, T_p, T_g, T_t, T_v, T_\mathbf{d}, T_w, T_r, T_s) & \text{否则} 
\end{cases}
\end{equation}
这里我们假设除了任意长度字符数组$T_\mathbf{i}$和$T_\mathbf{d}$外,所有部分都用RLP转换为整数值:
\begin{equation}
\begin{array}[t]{lclclc}
T_n \in \mathbb{P}_{256} & \wedge & T_v \in \mathbb{P}_{256} & \wedge & T_p \in \mathbb{P}_{256} & \wedge \\
T_g \in \mathbb{P}_{256} & \wedge & T_w \in \mathbb{P}_5 & \wedge & T_r \in \mathbb{P}_{256} & \wedge \\
T_s \in \mathbb{P}_{256} & \wedge & T_\mathbf{d} \in \mathbb{B} & \wedge & T_\mathbf{i} \in \mathbb{B}
\end{array}
\end{equation}
其中:
\begin{equation}
\mathbb{P}_n = \{ P: P \in \mathbb{P} \wedge P < 2^n \}
\end{equation}

地址散列$T_\mathbf{t}$略有不同:它要么是个20位的地址哈希;要么对于合约生成交易,是个空字符序列的RPL值(等于$\varnothing$),也就是$\mathbb{B}_0$的唯一成员:
\begin{equation}
T_t \in \begin{cases} \mathbb{B}_{20} & \text{如} \quad T_t \neq \varnothing \\
\mathbb{B}_{0} & \text{否则}\end{cases}
\end{equation}

\subsection{区块} \label{ch:block}

区块中包含由相关信息组成的区块头(block header)$H$,包含交易$\mathbf{T}$相关的信息,还包含一些其它区块的头$\mathbf{U}$,这些区块的父亲和这个区块父亲的父亲是同一个区块,因此这些区块也被称为\textit{ommers}\footnote{\textit{ommer} is the most prevalent (not saying much) gender-neutral term to mean ``sibling of parent''; see \url{http://nonbinary.org/wiki/Gender_neutral_language#Family_Terms}}。

区块头包括下面几种信息:

%\textit{TODO: Introduce logs}

\begin{description}
\item[parentHash] 父块散列,父区块头的256位Keccak散列值,表示为$H_p$。
\item[ommersHash] 叔父散列,该区块ommer列表的256位Keccak散列值,表示为$H_o$。
\item[beneficiary] 该区块挖矿所得费用的受益方的160位地址,表示为$H_c$。
\item[stateRoot] 状态根散列,在所有交易被成功执行后,状态树根节点的256位Keccak散列值,表示为$H_r$。
\item[transactionsRoot] 交易根散列,将该区块中交易列表中的每个交易填充到一个Trie树后,该Trie树根节点的256位Keccak散列值,表示为$H_t$。
\item[receiptsRoot] 收据根散列,将该区块中交易列表中的每个交易的收据填充到一个Trie树后,该Trie树根节点的256位Keccak散列值,表示为$H_e$。
\item[logsBloom] 由每个交易收据日志条目中所包含的可索引信息 --- 日志地址(logger address)和日志话题(log topics)--- 所构成的Bloom filter,表示为$H_b$。
\item[difficulty] 难度,是一个代表该区块的难度级别的标量。它的值可以根据前一个块的难度和当前块的时间戳(tiemstamp)计算得来,表示为$H_d$。
\item[number] 是一个标量,代表该区块一共有多少个祖先块,表示为$H_i$。创世纪区块的number是0。
\item[gasLimit] 是一个标量,代表目前每个区块gas花销的上限,表示为$H_l$。
\item[gasUsed] 是一个标量,其值是该区块中所有交易花费gas的总量,表示为$H_g$。
\item[timestamp] 时间戳,是一个标量,其值是该区块开始时Unix系统time()方法调用的返回值,表示为$H_s$。
\item[extraData] 和该区块相关的数据,是一个不超过32字节的字符数组,表示为$H_x$。
\item[mixHash] 一个256位的散列值,它和nonce一起使用时可用来证明该区块上承载了足够的工作量,表示为$H_m$。
\item[nonce] 临造数,是一个64位的散列值,它和mix-hash一起使用时可用来证明该区块上承载了足够的工作量,表示为$H_n$。
\end{description}

区块中还有两部分,一个是ommer区块头的列表(格式和上面的一样),另一个是一系列的交易。我们形式上定义区块$B$为:

\begin{equation}
B \equiv (B_H, B_\mathbf{T}, B_\mathbf{U})
\end{equation}

\subsubsection{交易收据}

[In order to encode information about a transaction concerning which it may be useful to form a zero-knowledge proof, or index and search, we encode a receipt of each transaction containing certain information from concerning its execution. Each receipt, denoted $B_\mathbf{R}[i]$ for the $i$th transaction) is placed in an index-keyed trie and the root recorded in the header as $H_e$.]

交易收据是包括下列信息的四元组:该交易执行后的状态$R_{\boldsymbol{\sigma}}$,该交易刚刚执行后该区块累计消耗的gas总量$R_u$,该交易执行过程中生成的日志集合$R_\mathbf{l}$,以及这些日志构成的Bloom filter $R_b$:
\begin{equation}
R \equiv (R_{\boldsymbol{\sigma}}, R_u, R_b, R_\mathbf{l})
\end{equation}

函数 $L_R$ 简单地将交易收据转换为一个经RLP序列化后的字节数组:
\begin{equation}
L_R(R) \equiv (\mathtt{\small TRIE}(L_S(R_{\boldsymbol{\sigma}})), R_u, R_b, R_\mathbf{l})
\end{equation}

因此该交易执行后的状态$R_{\boldsymbol{\sigma}}$ 被转码成一个Trie树结构,而该树的根是才是上面四元组的第一个值。

我们断定累计消耗gas数量 $R_u$ 是个正整数,日志的Bloom filter $R_b$是个2048位(256字节)的散列值:
\begin{equation}
R_u \in \mathbb{P} \quad \wedge \quad R_b \in \mathbb{B}_{256}
\end{equation}

%Notably $B_\mathbf{T}$ does not get serialised into the block by the block preparation function $L_B$; it is merely a convenience equivalence.

日志集合$R_\mathbf{l}$是被分成多个段的,比如分成 $(O_0, O_1, ...)$格式。每个日志条目 $O$都是一个元组,由日志记录者地址$O_a$,一系列32字节的日志话题$O_\mathbf{t}$ ,和几个字节的数据$O_\mathbf{d}$组成:
\begin{equation}
O \equiv (O_a, ({O_\mathbf{t}}_0, {O_\mathbf{t}}_1, ...), O_\mathbf{d})
\end{equation}
\begin{equation}
O_a \in \mathbb{B}_{20} \quad \wedge \quad \forall_{t \in O_\mathbf{t}}: t \in \mathbb{B}_{32} \quad \wedge \quad O_\mathbf{d} \in \mathbb{B}
\end{equation}

我们如下定义一个256字节的Bloom filter函数$M$:
\begin{equation}
M(O) \equiv \bigvee_{t \in \{O_a\} \cup O_\mathbf{t}} \big( M_{3:2048}(t) \big)
\end{equation}

其中 $M_{3:2048}$ 是可应用在任意长度字节数组的一个2048位选3位的Bloom filter。具体的做法先计算该字节数组的Keccak-256散列值,然后将这个散列值的前三个字节对的地位11个byte对应位置的bit设为1。【译者:每个字节对是$2\times 8 = 16$位,取低的11位最多可以表达$2^{11} = 2048$个位置信息,刚好对应于Bloom filter的长度。】  $M_{3:2048}$形式表达为:

\begin{eqnarray}
M_{3:2048}(\mathbf{x}: \mathbf{x} \in \mathbb{B}) & \equiv & \mathbf{y}: \mathbf{y} \in \mathbb{B}_{256} \quad \text{where:}\\
\mathbf{y} & = & (0, 0, ..., 0) \quad \text{except:}\\
\forall_{i \in \{0, 2, 4\}}&:& \mathcal{B}_{m(\mathbf{x}, i)}(\mathbf{y}) = 1\\
m(\mathbf{x}, i) &\equiv& \mathtt{\tiny KEC}(\mathbf{x})[i, i + 1] \bmod 2048
\end{eqnarray}

这里$\mathcal{B}$ 是个位引用函数,$\mathcal{B}_j(\mathbf{x})$代表字符数组$\mathbf{x}$的第$j$个位置(从0开始)的值。

\subsubsection{整体合法性}

当且仅当一个区块满足下列条件时我们才肯定它是有效的:它必须与所有叔父区块的散列值和所有交易(对应\ref{ch:finalisation}章节的描述)的散列值具有内在一致性,当在(父亲区块的最终状态演化而来的)基础状态$\boldsymbol{\sigma}$之上按照顺序执行的时候,必须得到一个可用头$H_r$代表的一个新状态:
\begin{equation}
\begin{array}[t]{lclc}
H_r &\equiv& \mathtt{\small TRIE}(L_S(\Pi(\boldsymbol{\sigma}, B))) & \wedge \\
H_o &\equiv& \mathtt{\small KEC}(\mathtt{\small RLP}(L_H^*(B_\mathbf{U}))) & \wedge \\
H_t &\equiv& \mathtt{\small TRIE}(\{\forall i < \lVert B_\mathbf{T} \rVert, i \in \mathbb{P}: p(i, L_T(B_\mathbf{T}[i]))\}) & \wedge \\
H_e &\equiv& \mathtt{\small TRIE}(\{\forall i < \lVert B_\mathbf{R} \rVert, i \in \mathbb{P}: p(i, L_R(B_\mathbf{R}[i]))\}) & \wedge \\
H_b &\equiv& \bigvee_{\mathbf{r} \in B_\mathbf{R}} \big( \mathbf{r}_b \big)
\end{array}
\end{equation}

这里 $p(k, v)$ 是个简单数据对维度的RLP转换 --- 第一个数据是该区块交易的序号,第二个数据是交易的收据:
\begin{equation}
p(k, v) \equiv \big( \mathtt{\small RLP}(k), \mathtt{\small RLP}(v) \big)
\end{equation}

另外:
\begin{equation}
\mathtt{\small TRIE}(L_S(\boldsymbol{\sigma})) = {P(B_H)_H}_r
\end{equation}

因此 $\texttt{\small TRIE}(L_S(\boldsymbol{\sigma}))$ 是状态树根节点散列值, $P(B_H)$被定义为区块$B$的父区块。

在\ref{sec:statenoncevalidation}章节我们正式定义由交易计算出来的一些值,特别是交易收据$B_\mathbf{R}$,以及通过交易状态收集函数(transactions state-accumulation function) $\Pi$计算出来的一些值。

\subsubsection{序列化}

$L_B$和$L_H$对应区块和区块头的准备函数。和交易收据准备函数非常类似,当需要对这两个函数做RLP转换的时候,我们对其类型和结构定义如下:
\begin{eqnarray}
\quad L_H(H) & \equiv & (\begin{array}[t]{l}H_p, H_o, H_c, H_r, H_t, H_e, H_b, H_d,\\ H_i, H_l, H_g, H_s, H_x, H_m, H_n \; )\end{array} \\
\quad L_B(B) & \equiv & \big( L_H(B_H), L_T^*(B_\mathbf{T}), L_H^*(B_\mathbf{U}) \big)
\end{eqnarray}

$L_T^*$ 是 $L_H^*$ 元素维度的序列转换,因此:
\begin{equation}
f^*\big( (x_0, x_1, ...) \big) \equiv \big( f(x_0), f(x_1), ... \big) \quad \text{对任意函数} \; f
\end{equation}

各部分类型定义如下:
\begin{equation}
\begin{array}[t]{lclclcl}
H_p \in \mathbb{B}_{32} & \wedge & H_o \in \mathbb{B}_{32} & \wedge & H_c \in \mathbb{B}_{20} & \wedge \\
H_r \in \mathbb{B}_{32} & \wedge & H_t \in \mathbb{B}_{32} & \wedge & H_e \in \mathbb{B}_{32} & \wedge \\
H_b \in \mathbb{B}_{256} & \wedge & H_d \in \mathbb{P} & \wedge & H_i \in \mathbb{P} & \wedge \\
H_l \in \mathbb{P} & \wedge & H_g \in \mathbb{P} & \wedge & H_s \in \mathbb{P}_{256} & \wedge \\
H_x \in \mathbb{B} & \wedge & H_m \in \mathbb{B}_{32} & \wedge & H_n \in \mathbb{B}_{8}
\end{array}
\end{equation}

其中:
\begin{equation}
\mathbb{B}_n = \{ B: B \in \mathbb{B} \wedge \lVert B \rVert = n \}
\end{equation}

现在我们有了一个区块结构的严谨规范。RLP函数$\texttt{\small RLP}$(参见附录\ref{app:rlp})提供区块序列化成字节数组的标准方法,便于数据通过网络传输或者用于读写本地存储。
\subsubsection{区块头的合法性}

我们定义区块$B$的父区块为$P(B_H)$:
\begin{equation}
P(H) \equiv B': \mathtt{\tiny KEC}(\mathtt{\tiny RLP}(B'_H)) = H_p
\end{equation}

区块的序号是父区块序号加1得来:
\begin{equation}
H_i \equiv {{P(H)_H}_i} + 1
\end{equation}

\newcommand{\mindifficulty}{D_0}
\newcommand{\frontiermod}{\ensuremath{\varsigma_1}}
\newcommand{\homesteadmod}{\ensuremath{\varsigma_2}}
\newcommand{\expdiffsymb}{\ensuremath{\epsilon}}
\newcommand{\diffadjustment}{x}

区块头$H$的标准难度$D(H)$定义为:
\begin{equation}
D(H) \equiv \begin{dcases}
\mindifficulty & \text{如} \quad H_i = 0\\
\text{max}\!\left(\mindifficulty, {P(H)_H}_d + \diffadjustment\times\frontiermod + \expdiffsymb \right) & \text{如} \quad H_i<\firsthomesteadblock\\
\text{max}\!\left(\mindifficulty, {P(H)_H}_d + \diffadjustment\times\homesteadmod + \expdiffsymb \right) & \text{否则}\\
\end{dcases}
\end{equation}
其中:
\begin{equation}
\mindifficulty \equiv 131072
\end{equation}
\begin{equation}
\diffadjustment \equiv \left\lfloor\frac{{P(H)_H}_d}{2048}\right\rfloor
\end{equation}
\begin{equation}
\frontiermod \equiv \begin{cases}
1 & \text{如} \quad H_s < {P(H)_H}_s + 13 \\
-1 & \text{否则} \\
\end{cases}
\end{equation}
\begin{equation}
\homesteadmod \equiv \text{max}\left( 1 - \left\lfloor\frac{H_s - {P(H)_H}_s}{10}\right\rfloor, -99 \right)
\end{equation}
\begin{equation}
\expdiffsymb \equiv \left\lfloor 2^{ \left\lfloor H_i \div 100000 \right\rfloor - 2 } \right\rfloor
\end{equation}

区块头$H$的gasLimit$H_l$必须满足:
\begin{eqnarray}
& & H_l < {P(H)_H}_l + \left\lfloor\frac{{P(H)_H}_l}{1024}\right\rfloor \quad \wedge \\
& & H_l > {P(H)_H}_l - \left\lfloor\frac{{P(H)_H}_l}{1024}\right\rfloor \quad \wedge \\
& & H_l \geqslant 125000
\end{eqnarray}

区块$H$的时间戳$H_s$必须满足:
\begin{equation}
H_s > {P(H)_H}_s
\end{equation}

这种机制确保了区块间时间间隙的动态平衡。如果最后两个区块间的间隙小,难度值和计算量就会增加,造成下一个区块的间隙的增加。相反,如果最后两个区块间的间隙太大,难度值和下一个块的时间间隙就会降低。

临造数$H_n$必须满足:
\begin{equation}
n \leqslant \frac{2^{256}}{H_d} \quad \wedge \quad m = H_m
\end{equation}
其中 $(n, m) = \mathtt{PoW}(H_{\hcancel{n}}, H_n, \mathbf{d})$.

这里 $H_{\hcancel{n}}$ 是新区块的头$H$,但\textit{不包含} nonce 和 mixHash两个域;$\mathbf{d}$ 是当前 DAG,它是用来计算mixHash的一个很大的数据集;$\mathtt{PoW}$ 是工作量证明函数(参考\ref{ch:pow}章节)。这会形成一个数组,它的第一个值是mixHash,用来证明使用的DAG是对的;它的第二个值是密码学上依赖于$H$和$\mathbf{d}$的一个伪随机数。由于在$[0, 2^{64})$区间的分布基本是均匀的,因此找到答案所需要的时间和难度$H_d$成正比。

这就是区块链安全性的基础,也是为什么恶意节点无法在网络中传播可以篡(``改写'')历史的新区块的根本原因。由于nonce必须满足特定要求,而这种要求又依赖于区块中的内容,即构成该区块的所有交易,因此生成新的合法的区块是有难度的。从长远看,这么做需要恶意节点具有相当于挖矿资源中可信赖部分的全部的计算量。

因此我们可以这样定义区块头的验证函数$V(H)$:

\begin{eqnarray}
V(H) & \equiv &  n \leqslant \frac{2^{256}}{H_d} \wedge m = H_m \quad \wedge \\
& & H_d = D(H) \quad \wedge \\
& & H_g \le H_l  \quad \wedge \\
& & H_l < {P(H)_H}_l + \left\lfloor\frac{{P(H)_H}_l}{1024}\right\rfloor  \quad \wedge \\
& & H_l > {P(H)_H}_l - \left\lfloor\frac{{P(H)_H}_l}{1024}\right\rfloor  \quad \wedge \\
& & H_l \geqslant 125000  \quad \wedge \\
& & H_s > {P(H)_H}_s \quad \wedge \\
& & H_i = {P(H)_H}_i +1 \quad \wedge \\
& & \lVert H_x \rVert \le 32
\end{eqnarray}
这里 $(n, m) = \mathtt{PoW}(H_{\hcancel{n}}, H_n, \mathbf{d})$

另外需要注意\textbf{extraData}不能超过32个字节。
%-------------------------------------

\section{油费和付款} \label{ch:payment}

为了避免网络被乱用,规避图灵完备带来的不可避免的一些问题,以太繁重所有的可编程计算都是收费的。我们用\textit{gas} (参见附录\ref{app:fees}中各种计算的费用)作为费用的衡量标准。因此任何一段可编程计算(包括合约生成,消息调用,读取和使用账户存储空间,以及在虚拟机上执行的操作)都有一个被公认的gas消耗量。

每个交易都有一个与gas相关的特定值:\textbf{gasLimit}。这是值是用发送者账户余额自动购买的gas的数量。购买时使用\textbf{gasPrice}指定的价格,该价格也同样包含在这个交易中。如果账户余额不足,该交易就被认定是无效的。之所以被称为\textbf{gasLimit},是因为如果交易执行完毕后如果还有未用完的gas,这些剩余的gas会被(按原价)退回,得到的以太会退款到发送者的账户。在交易执行之外gas是不存在的。因此对于那些有可信代码的账户,设置一个较高的gas上限并不会有什么问题。

总的来说,用来购买gas的以太如果不被退款,就会被转给一个受益人(\textit{beneficiary})地址 - 这个地址一般是旷工控制的账户的地址。交易人可以自由设定想用的\textbf{gasPrice} ,但旷工也可以自由决定忽略哪些交易。高油费的交易会消耗发送者更多的以太,但对旷工也更有价值,因此被更多旷工放到区块中的可能性就更大。一般来说旷工会公布他们执行交易是愿意接受的最低油价,而交易者在选择油价的时候也可以询问旷工的公布的价格。由于矿工们可接受的最低油价是个(加权的)分布,交易人需要在降低油价和最大化自己的交易被及时挖到之间做个权衡。

%\subsubsection{Determining Computation Costs}

%-------------------------------------

\section{交易执行} \label{ch:transactions}

交易执行是以太坊协议最复杂的部分:它定义了状态转换函数 $\Upsilon$。任何交易的执行都需要先通过初始状态的内在合法性验证。这包括:

\begin{enumerate}
\item 交易以正确的RLP形式表示,没有多余字节跟在后面;
\item 交易的签名正确;
\item 交易的nonce正确(和发送账号当前nonce相同);
\item gasLimit不能小于交易用到的gas数量$g_0$;
\item 发送者账户余额不小于需要预付的花费$v_0$。
\end{enumerate}

我们这样表达$\Upsilon$函数,其中$T$是交易,$\boldsymbol{\sigma}$是状态:
\begin{equation}
\boldsymbol{\sigma}' = \Upsilon(\boldsymbol{\sigma}, T)
\end{equation}

$\boldsymbol{\sigma}'$是交易执行后的状态。同时我们定义 $\Upsilon^g$ 为计算该交易执行所需gas的函数,$\Upsilon^\mathbf{l}$为计算该交易累计日志条目的函数。我们后续会给出正式定义。

\subsection{子状态} 
在交易执行过程中,我们会收集一些信息,这些信息在交易执行之后会被用到。我们称这些信息为\textit{子状态(transaction substate)}, 用$A$表示。子状态是个元组:
\begin{equation}
A \equiv (A_\mathbf{s}, A_\mathbf{l}, A_r)
\end{equation}

元组中包含一个自杀集$A_\mathbf{s}$,它是一组在交易完成后即会被丢弃的账户。$A_\mathbf{l}$是日志,是归档并索引过的虚拟机代码执行的`检查点(checkpoint)',检查点允许以太坊之外的观察者(比如去中心化应用的前端)方便地追踪合约的调用情况。最后是存储退款额(refund balance)$A_r$,使用{\small SSTORE} 指令释放合约存储空间后退款额会变大。存储退款额虽然不是立即生效,但可以降低合约执行过程中的总花费。

为了简化,我们定义$A^0$为空子状态,它没有自杀集,没有日志,并且退款额为零:
\begin{equation}
A^0 \equiv (\varnothing, (), 0)
\end{equation}

\subsection{执行}
我们这样定义交易执行前需要预先支付的gas的数量$g_0$:
\begin{align}
g_0 \equiv {} & \sum_{i \in T_\mathbf{i}, T_\mathbf{d}} \begin{cases} G_{txdatazero} & \text{如} \quad i = 0 \\ G_{txdatanonzero} & \text{否则} \end{cases} \\
{} & + \begin{cases} G_\text{txcreate} & \text{if} \quad T_t = \varnothing \wedge H_i \geq \firsthomesteadblock \\ 0 & \text{否则} \end{cases} \\
{} & + G_{transaction}
\end{align}

这里 $T_\mathbf{i},T_\mathbf{d}$ 是交易的EVM的初始代码或数据,取决于该交易是合约生成类型的交易还是消息调用类型的交易。如果是合约生成类型的交易,$G_\text{txcreate}$会被累加上去;但如果交易不是合约生成型的,或者交易是在{\it 过度到家园(Homestead)分叉}之前执行的,$G_\text{txcreate}$则不会被累加上去。在附录\ref{app:fees}对$G$做了完整定义。

%todo Explain g_d reason?

需要预付的花费 $v_0$定义如下:
\begin{equation}
v_0 \equiv T_g T_p + T_v
\end{equation}

正确性的定义如下:
\begin{equation}
\begin{array}[t]{rcl}
S(T) & \neq & \varnothing \quad \wedge \\
\boldsymbol{\sigma}[S(T)] & \neq & \varnothing \quad \wedge \\
T_n & = & \boldsymbol{\sigma}[S(T)]_n \quad \wedge \\
g_0 & \leqslant & T_g \quad \wedge \\ 
v_0 & \leqslant & \boldsymbol{\sigma}[S(T)]_b \quad \wedge \\
T_g & \leqslant & {B_H}_l - \ell(B_\mathbf{R})_u
\end{array}
\end{equation}

注意最后一个条件:该交易的油费上限$T_g$加上该区块之前所有油费消耗之和$\ell(B_\mathbf{R})_u$,不能大于该区块的 \textbf{gasLimit} ${B_H}_l$。

一个合法交易的执行伴随着对状态的不可撤销的变更:发送者账户$S(T)$的nonce被加一,账户余额被减去预付款额$T_gT_p$。可用于继续计算的油费$g$等于$T_g - g_0$。无论是合约生成还是消息调用,计算都会结束于一个最终状态(有可能和当前状态一样,这也是合法的)。状态更改是确定性的且保证是合法的:后续不会因此而存在非法交易。

我们定义检查点状态(checkpoint state) $\boldsymbol{\sigma}_0$为:
\begin{eqnarray}
\boldsymbol{\sigma}_0 & \equiv & \boldsymbol{\sigma} \quad \text{但是} \\
\boldsymbol{\sigma}_0[S(T)]_b & \equiv & \boldsymbol{\sigma}[S(T)]_b - T_g T_p \\
\boldsymbol{\sigma}_0[S(T)]_n & \equiv & \boldsymbol{\sigma}[S(T)]_n + 1
\end{eqnarray}

我们用$\boldsymbol{\sigma}_P$表示交易后的临时状态。基于$\boldsymbol{\sigma}_0$去计算$\boldsymbol{\sigma}_P$取决于该交易是合约生成还是消息调用。我们定义交易后临时状态$\boldsymbol{\sigma}_P$,剩余油费$g'$和子状态 $A$构成的三元组为:
\begin{equation}
(\boldsymbol{\sigma}_P, g', A) \equiv \begin{cases}
\Lambda(\boldsymbol{\sigma}_0, S(T), T_o, &\\ \quad\quad g, T_p, T_v, T_\mathbf{i}, 0) & \text{如} \quad T_t = \varnothing \\
\Theta_{3}(\boldsymbol{\sigma}_0, S(T), T_o, &\\ \quad\quad T_t, T_t, g, T_p, T_v, T_v, T_\mathbf{d}, 0) & \text{否则}
\end{cases}
\end{equation}

其中$g$是扣除合约存在所需支付的基本花费后的剩余油费:
\begin{equation}
g \equiv T_g - g_0
\end{equation}

$T_o$表示原始交易人(original transactor)。如果消息调用或合约生成不是通过交易出发,而是通过EVM代码的执行而触发的,原始交易人和发送者(sender)就不相同。

注意,我们用$\Theta_{3}$表示方法值中只有前三个被使用。[不知道如何翻译:the final represents the message-call's output value (a byte array) and is unused in the context of transaction evaluation.]

消息调用或者合约生成处理完成后,会从剩余油费$g'$中用同样的价格计算退出款额$g^*$,并累加上退款计数器中的数值。
\begin{equation}
g^* \equiv g' + \min \{ \Big\lfloor \dfrac{T_g - g'}{2} \Big\rfloor, A_r \}
\end{equation}

退款总额是剩余油费$g'$加上存储退款额$A_r$,但$A_r$被限制到不能超过实际油费消耗$T_g - g'$的一半。

购买油费所用的以太会被奖励给旷工,旷工由当前块$B$的受益人所指定。我们通过临时状态$\boldsymbol{\sigma}_P$定以终前状态(pre-final state)$\boldsymbol{\sigma}^*$:
\begin{eqnarray}
\boldsymbol{\sigma}^* & \equiv & \boldsymbol{\sigma}_P \quad \text{但是} \\
\boldsymbol{\sigma}^*[S(T)]_b & \equiv & \boldsymbol{\sigma}_P[S(T)]_b + g^* T_p \\
\boldsymbol{\sigma}^*[m]_b & \equiv & \boldsymbol{\sigma}_P[m]_b + (T_g - g^*) T_p \\
m & \equiv & {B_H}_c
\end{eqnarray}

删除自杀集中的所有账户后,就得到了最终状态$\boldsymbol{\sigma}'$:
\begin{eqnarray}
\boldsymbol{\sigma}' & \equiv & \boldsymbol{\sigma}^* \quad \text{但是} \\
\forall i \in A_\mathbf{s}: \boldsymbol{\sigma}'[i] & \equiv & \varnothing
\end{eqnarray}

最后我们定义该交易的耗油总量$\Upsilon^g$,以及该交易生成的日志$\Upsilon^\mathbf{l}$为:
\begin{eqnarray}
\Upsilon^g(\boldsymbol{\sigma}, T) & \equiv & T_g - g' \\
\Upsilon^\mathbf{l}(\boldsymbol{\sigma}, T) & \equiv & A_\mathbf{l}
\end{eqnarray}

它们可以帮助定义我们后续要讲的交易收据。

%In the case that $s = m$ then we simply return the Ether back to the sender/miner, collapsing the exception into:
%\begin{eqnarray}
%\boldsymbol{\sigma}'[s]_b & \equiv & \boldsymbol{\sigma}_P[s]_b + g
%\end{eqnarray}


%-------------------------------------

\section{合约生成} \label{ch:create}

生成账户时需要用到几个固定的参数:发送者(sender,$s$),原始交易者(original transactor,$o$),可用油费($g$),油价($p$),endowment ($v$),一个包含EVM初始化代码的任意长度的字节数组$\mathbf{i}$,和当前消息调用/合约生成栈的深度($e$)。

我们正式定义生成函数为$\Lambda$。生成函数的计算除了需要上述参数,还需要状态 $\boldsymbol{\sigma}$,以及我们在\ref{ch:transactions}章节描述的,一个包括新状态,剩余油费,和交易子状态的三元组$(\boldsymbol{\sigma}', g', A)$:

\begin{equation}
(\boldsymbol{\sigma}', g', A) \equiv \Lambda(\boldsymbol{\sigma}, s, o, g, p, v, \mathbf{i}, e)
\end{equation}

新账户$a$的地址被定为发送者(sender)和nonce经过RLP编码后的Keccak散列值的最右160位:
\begin{equation}
a \equiv \mathcal{B}_{96..255}\Big(\mathtt{\tiny KEC}\Big(\mathtt{\tiny RLP}\big(\;(s, \boldsymbol{\sigma}[s]_n - 1)\;\big)\Big)\Big)
\end{equation}

这里 $\mathtt{\tiny KEC}$ 是256位Keccak散列函数,$\mathtt{\tiny RLP}$是RLP编码函数,
$\mathcal{B}_{a..b}(X)$计算出二进制数据$X$在$[a, b]$索引区间的比特值;$\boldsymbol{\sigma}[x]$是$x$的地址状态(不存在则为$\varnothing$)。注意我们用的nonce值比发送者的nonce值小1;我们确信在调用该方法前发送者账户的nounce已经加1了,因此我们用的值是交易或者虚拟机操作之初发送者的nonce值。

生成的新账户nonce初始值为零,余额等于传入参数的值,存储空间为空,代码哈希为空字符串的Keccak 256位散列值;同时发送者账户余额会被扣掉传入参数的值。 因此变化后的状态为 $\boldsymbol{\sigma}^*$:
\begin{equation}
\boldsymbol{\sigma}^* \equiv \boldsymbol{\sigma} \quad \text{但是:}
\end{equation}
\begin{eqnarray}
\boldsymbol{\sigma}^*[a] &\equiv& \big( 0, v + v', \mathtt{\tiny TRIE}(\varnothing), \mathtt{\tiny KEC}\big(()\big) \big) \\
\boldsymbol{\sigma}^*[s]_b &\equiv& \boldsymbol{\sigma}[s]_b - v
\end{eqnarray}

其中 $v'$是该账户存在前的余额,如果它曾经存在过:
\begin{equation}
v' \equiv \begin{cases}
0 & \text{如果} \quad \boldsymbol{\sigma}[a] = \varnothing\\
\boldsymbol{\sigma}[a]_b & \text{否则}
\end{cases}
\end{equation}

%It is asserted that the state database will also change such that it defines the pair $(\mathtt{\tiny KEC}(\mathbf{b}), \mathbf{b})$.

最后,该账号的初始化代码会依照执行模型(参见\ref{ch:model})来对账号进行初始化。代码执行可以造成几个外在影响:该账号的存储空间可被更改,也可能生成更多账户,或发生更多的消息调用。正因如此,代码执行函数$\Xi$的结果是一个元组,包括结果状态$\boldsymbol{\sigma}^{**}$,可用gas余额$g^{**}$,累积子状态$A$,以及账号的主代码$\mathbf{o}$。

\begin{equation}
(\boldsymbol{\sigma}^{**}, g^{**}, A, \mathbf{o}) \equiv \Xi(\boldsymbol{\sigma}^*, g, I) \\
\end{equation}

其中$I$包含在\ref{ch:model}章节定义的运行环境参数,即:

\begin{eqnarray}
I_a & \equiv & a \\
I_o & \equiv & o \\
I_p & \equiv & p \\
I_\mathbf{d} & \equiv & () \\
I_s & \equiv & s \\
I_v & \equiv & v \\
I_\mathbf{b} & \equiv & \mathbf{i} \\
I_e & \equiv & e
\end{eqnarray}

因为该调用没有输入数据,$I_\mathbf{d}$的执行结果是空元组。$I_H$没有特殊处理并且由区块链决定。

代码执行消耗gas,但gas不可以小于零,因此代码执行在到达正常终结状态前就可能退出。在这种(以及其他一些)异常情况下,我们就说出现了Out-of-Gas(邮费耗尽)异常:结果状态被定义为空,$\varnothing$,并且整个账户生成操作对状态不应该有任何影响,状态和尝试账户生成之前的状态完全相同。

如果账号初始化代码执行成功,还需要支付一笔账户生成费用,即代码提交费$c$,它和生成账户主代码的大小成正比:

\begin{equation}
c \equiv G_{codedeposit} \times |\mathbf{o}|
\end{equation}

如果没有足够gas支付这笔费用,即 $g^{**} < c$,那么也会抛出Out-of-Gas异常。

在任何这种异常情况下,gas余额都会是零。比如,合约生成如果是由某个交易触发的,那么异常不会影响合约生成所需要的花费,该花费无论如何要付掉。尽管如此,当出现Out-of-Gas异常的时候,交易中的价值不会被转移到被中止生成的合约中。

如果没有出现这种异常,剩余的gas会被返还给发起者,已经更改的状态会得以保存。因此,我们可以将结果状态,gas,和子状态$(\boldsymbol{\sigma}', g', A)$这样定义:

\begin{align}
\quad g' &\equiv \begin{cases}
0 & \text{如果} \quad \boldsymbol{\sigma}^{**} = \varnothing \\
g^{**} & \text{如果} \quad g^{**}<c \wedge H_i<\firsthomesteadblock \\
g^{**} - c & \text{否则} \\
\end{cases} \\
\quad \boldsymbol{\sigma}' &\equiv  \begin{cases}
\boldsymbol{\sigma} & \text{如果} \quad \boldsymbol{\sigma}^{**} = \varnothing \\
\boldsymbol{\sigma}^{**} & \text{如果} \quad g^{**}<c \wedge H_i<\firsthomesteadblock \\
\boldsymbol{\sigma}^{**} \quad \text{但:} & \\
\quad\boldsymbol{\sigma}'[a]_c = \texttt{\small KEC}(\mathbf{o}) & \text{否则}
\end{cases}
\end{align}

计算$\boldsymbol{\sigma}'$公式中的例外情况,决定了初始化代码执行结果的字符序列其实就是新合约账户的主代码。

注意,从第\firsthomesteadblock\ 个块开始({\it Homestead}) ,账户生成的结果要么是一个附带捐款的合约,要么是一个没有捐款的新合约。在 {\it Homestead}之前,如果没有足够的gas来支付$c$,新合约账号也会被生成,初始代码也会被执行并产生影响,捐款也会被转账,但合约主代码不会被提交存储。

\subsection{几处细节}
注意,在初始化代码执行过程中,新地址是存在的,但它还没有主代码(body code)。因此它在此期间接收到的任何消息调用都不会造成代码被运行。如果初始化代码执行的结果是一个{\small SUICIDE}指令,就会造成交易执行完之前该账户就被删掉了,交易也就没有意义了。如果代码执行结果是{\small STOP}指令,或者返回结果是空值,那么就会留下一个僵尸账户,它的任何余额都会被永久锁住。

%-------------------------------------
%-------------------------------------
%-------------------------------------

\bibliography{Biblio}
\bibliographystyle{plainnat}

\end{multicols}
\end{document}
